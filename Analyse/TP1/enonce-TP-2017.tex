\documentclass[a4paper,11pt]{article}

\usepackage[utf8x]{inputenc}
\usepackage[T1]{fontenc}
\usepackage[francais]{babel}
\usepackage{amsmath,amssymb}
\usepackage{fullpage}
\usepackage{xspace}
\usepackage{verbatim}
\usepackage{graphicx}
\usepackage{listings}
\usepackage[usenames,dvipsnames]{color}
\usepackage{url}

\lstset{basicstyle=\small\tt,
  keywordstyle=\bfseries\color{Orchid},
  stringstyle=\it\color{Tan},
  commentstyle=\it\color{LimeGreen},
  showstringspaces=false}

\newtheorem{question}{Question}
\newtheorem{exo}{Exercice}

\newcommand{\dx}{\,dx}
\newcommand{\ito}{,\dotsc,}
\newcommand{\R}{\mathbb{R}}
\newcommand{\C}{\mathbb{C}}
\newcommand{\N}{\mathbb{N}}
\newcommand{\Poly}[1]{\mathcal{P}_{#1}}
\newcommand{\abs}[1]{\left\lvert#1\right\rvert}
\newcommand{\norm}[1]{\left\lVert#1\right\rVert}
\newcommand{\pars}[1]{\left(#1\right)}
\newcommand{\bigpars}[1]{\bigl(#1\bigr)}
\newcommand{\set}[1]{\left\{#1\right\}}

\title{Stage de Toussaint Ensimag 1A \\
  Mini-stage Scilab/Latex}
\author{Alexis Arnaud \and Amélie Fondevilla \and Li Liu \and Duong Hung Pham}
\date{Novembre 2017}

% ===============
\begin{document}
% ===============
\maketitle

\noindent Les objectifs de ce mini-stage sont les suivants:
\begin{enumerate}
\item Acquérir la maîtrise de {\bf Scilab}, qui est un outil de calcul
  numérique.
\item Acquérir la maîtrise de {\bf \LaTeX}, qui est un outil de
  traitement de texte, utilisé par les scientifiques (utile notamment
  pour les formules mathématiques).
\item Apprendre à rédiger un compte-rendu de TP, à partir de quelques
  exercices.
\end{enumerate}
Tous les documents pour ce TP sont disponibles sur chamilo
 dans le dossier :\\
\begin{center}
ENSIMAG3MMAI/Documents/TP Scilab-Latex
\end{center}

La partie 1 de ce sujet vous propose une introduction rapide à
Scilab. Prenez le temps de répondre aux questions et de feuilleter le
polycopié. Familiarisez-vous ensuite à \LaTeX\ avec la
partie 2, le polycopié et le fichier exemple fourni enonce.tex. Enfin, faites les
exercices de la partie TP, que vous rédigerez dans un compte-rendu.
Consignes pour le compte-rendu:
\begin{itemize}
\item Travailler en binôme.
\item Rendre une archive .tar.gz par Teide, et contenant le
  compte-rendu au format pdf ainsi que le code au format sce. Votre
  compte-rendu donnera lieu à un bonus/malus pour la note d'analyse.
\item Ne pas mettre le code dans le compte-rendu.
\item Être concis: quelques pages doivent suffire!
\item Se relire, soigner les figures et les formules.
\item[]
\end{itemize}
\begin{center}
\textbf{Date limite de rendu: 22 novembre 2017}\\
\end{center}

Bon courage à tous, et n'hésitez pas à faire appel aux encadrants
aussi souvent que nécessaire!


%======================================
\section{Prise en main de Scilab }
%======================================

\subsection{Introduction}

Pour lancer le logiciel Scilab il faut taper dans un terminal \verb+scilab &+. Une fenêtre de commandes
s'ouvre.  Après avoir lancé Scilab, vous pouvez tester les commandes
suivantes:
\begin{itemize}
\item \verb+help+ pour ouvrir l'aide en Scilab,
\item \verb+help mot-clé+ pour obtenir la description de la fonction
    \verb+mot-clé+,
\item \verb+apropos mot-clé+ pour obtenir la liste des pages d'aide
    contenant \verb+mot-clé+,
\item \verb+quit+ pour sortir de Scilab.
\end{itemize}

\begin{question} Donner la description de la commande {\tt spec}.
  Trouver la commande Scilab qui permet de définir la matrice identité.
\end{question}

Les commandes \verb+clear+, \verb+clc+ et \verb+clf+ permettent
d'effacer respectivement les données mises en mémoire, l'écran de
commandes et les figures. Elles doivent être exécutées régulièrement
pour éviter les erreurs et libérer la mémoire.

\subsection{Exécuter sous Scilab}

Les commandes Scilab peuvent être tapées directement en ligne, par
exemple:
\begin{verbatim}
--> A = ones(3,4)
\end{verbatim}
ou bien, écrites dans un fichier de commandes ``{\tt *.sce}''. Nous
vous conseillons d'utiliser un éditeur de texte qui offre la
coloration syntaxique pour le language Scilab; par exemple, {\it
  SciNotes}, accessible depuis le menu ``applications'' de
Scilab. Pour cela,
\begin{enumerate}
\item ouvrir un fichier intitulé par exemple {\tt test.sce},
  comportant les lignes suivantes:
  \begin{lstlisting}[language=scilab]
    // Ceci est un commentaire
    clc; clf; clear;
    A = ones(3, 4)
  \end{lstlisting}
\item sous Scilab, taper:
\begin{verbatim}
--> exec("test.sce")
\end{verbatim}
\end{enumerate}

Si l'on veut définir une fonction, par exemple la fonction {\it carré},
on peut le faire soit dans notre script {\tt tests.sce}, soit dans un
autre fichier, par exemple {\tt carre.sce}:
\begin{enumerate}
\item Ouvrir un fichier intitulé par exemple {\tt carre.sce},
  comportant les instructions suivantes:
  \begin{lstlisting}[language=scilab]
    //Fonction carre
    function d = carre(x)
       d = x .* x
    endfunction
  \end{lstlisting}

\item Sous Scilab, charger le fichier {\tt carre.sce}:
\begin{verbatim}
--> exec("carre.sce")
\end{verbatim}
\end{enumerate}
La fonction {\it carré} est maintenant définie sous Scilab:
\begin{verbatim}
--> x = [0, 1, 2, 3, 4]
--> carre(x)
\end{verbatim}

\subsection{Vecteurs et matrices}

Scilab est un logiciel de calcul \emph{numérique}, l'essentiel des objets que l'on manipule sont des matrices de nombre.

La façon la plus simple de définir une matrice $n \times m$ en Scilab
est d'entrer au clavier la liste de ses éléments, par exemple
\begin{verbatim}
--> A = [a11, a12, a13; a21, a22, a23]
\end{verbatim}

\paragraph{Opérations élémentaires} A tester sur des exemples!
\begin{verbatim}
--> A(k,:)  // k-ième ligne de la matrice A
--> A + B   // somme
--> A * B   // produit
--> A .* B  // produit terme à terme
--> A^2     // équivalent à A * A
--> A.^2    // équivalent à A .* A
--> det(A)  // déterminant de A
--> A'      // transposée de A
--> inv(A)  // inverse de A
\end{verbatim}

\begin{question} Entrer sous Scilab la matrice:
  \begin{equation*}
    A = \left( \begin{array}{cccc}
      1 & 0 & 3 & 1 \\
      1 & 2 & 0 & 1 \\
      0 & 1 & 3 & 0
    \end{array} \right).
  \end{equation*}
  \begin{enumerate}
  \item Quelle est la commande donnant la taille de la matrice A ?
  \item Extraire la première ligne, la dernière colonne et l'élément à
    la deuxième ligne, troisième colonne de $A$.
  \item Extraire la diagonale, les parties triangulaires supérieures et
    inférieures de $A$.
  \end{enumerate}
\end{question}

\begin{question}[Matrices particulières]~
  \begin{enumerate}
  \item En utilisant les fonctions {\tt ones} et {\tt diag}, définir la
    matrice identité $10\times 10$.
  \item Définir la matrice tridiagonale d'ordre $10$ suivante en utilisant ces
    commandes:
    \begin{equation*}
      C = \left(\begin{array}{cccc}
          2 & -1& & \mathbf{0}\\
          1 & 2 & \ddots & \\
          & \ddots & \ddots &-1\\
          \mathbf{0}& & 1 &2 \\
        \end{array} \right).
    \end{equation*}
  \end{enumerate}
\end{question}


\subsection{Fonctions}

\subsubsection{Fonctions échantillonnées (= discrétisées)}

Une fonction peut être définie par rapport à une discrétisation de la
variable $x$, ainsi:
\begin{verbatim}
--> x = 0:0.1:1
\end{verbatim}
correspond à une discrétisation par pas de 0.1, de l'intervalle
$[0,1]$, soit 11 valeurs. On définit des fonctions sur cette grille
discrète, par exemple:
\begin{verbatim}
--> y = sin(2 * %pi * x) + cos(%pi * x)  // somme de deux sinusoïdes
--> z = x.^2                             // parabole
\end{verbatim}

\noindent {\it Remarque :} Les fonctions prédéfinies (comme {\tt cos} ou {\tt
  sin}) agissent terme à terme sur les vecteurs, et même sur les
matrices. Ainsi on n'a pas besoin de faire de boucles. En règle
générale on évite les boucles autant que possible, car elles
ralentissent l'exécution.

\subsubsection{Tracé de courbes}

Pour tracer une courbe $y=x^2$ sur l'intervalle $[a,b]$:
\begin{verbatim}
--> n = ...              // nombre de points de discrétisation
--> dx = (b - a)/(n - 1) // pas de la discrétisation
--> x = a:dx:b;          // x est échantillonné entre a et b avec un pas de dx
--> plot(x, x.^2)
\end{verbatim}
{\it Remarques :}
\begin{itemize}
\item On pourra également utiliser la commande {\tt x=linspace(a,b,n)}
  pour définir la grille discrète.
\item La commande {\tt plot} est une fonction de Matlab qui a ensuite
  été intégrée à Scilab. On peut aussi utiliser la fonction originale de
  Scilab {\tt plot2d} décrite dans le poly.
\end{itemize}
~
\begin{description}
\item[Pour varier le trait ou la couleur] ~
\begin{verbatim}
--> plot(x, f(x), "r-")
\end{verbatim}
  Dans la chaîne de caractère (troisième argument), on donne une
  lettre imposant la couleur ({\tt "r"} pour red, {\tt "b"} pour blue,
  {\tt "g"} pour green...) et un symbole pour le trait ({\tt "-"} pour un
  trait continu, {\tt "-\,-"} pour des tirets, {\tt "-."} pour des
  pointillés...). Pour plus de détails, {\tt help LineSpec}.

  ~
\item[Pour tracer plusieurs courbes, rajouter un titre, des axes, une
  légende] ~
\begin{verbatim}
--> x = 0:0.1:10
--> plot(x, cos(x), "r", x, sin(x), "b")
--> xtitle("Graphe de la fonction sin")  // titre
--> legend("cos","sin")                  // légende
--> xlabel("temps"); ylabel("f(t)");     // noms des axes
\end{verbatim}

  ~
\item[Pour tracer plusieurs graphes dans une fenêtre] ~\\
  La commande {\tt subplot(n, m, p)} placée avant chaque tracé de
  courbe, subdivise la fenêtre du graphe en une matrice $n\times m$ de
  sous-fenêtres et sélectionne la $p$-ième pour dessiner le graphe
  courant: l'élément $(i, j)$ de la matrice correspond au graphe numéro
  $(i-1)n + j$.

  ~
\item[Pour créer une nouvelle figure] ~\\
  Taper {\tt figure()} par exemple pour créer une nouvelle fenêtre.

  ~
\item[Pour exporter une figure] ~\\
  Dans la fenêtre graphique à exporter, cliquer sur le menu {\tt
    File}, puis {\tt Export}. Dans la fenêtre qui s'ouvre alors:
  \begin{enumerate}
  \item choisir l'extension du fichier image: choisir de préférence {\tt pdf},
  \item choisir entre {\tt Color} et {\tt Black \& White},
  \item choisir l'orientation. Attention, par défaut, l'image est en
    landscape (format paysage)!
  \item entrer le nom du fichier image: {\tt myfig} par exemple, sans
    l'extension.
  \end{enumerate}
\end{description}

\newpage

\begin{question} 
  Tracer les courbes de la fonction {\it sin} sur l'intervalle
  $[0,2\pi]$ pour 6 points de discrétisation, puis 21 points de
  discrétisation (en une autre couleur). Ajouter un titre et une légende.
\end{question}


%=====================================
\section{Prise en main de Latex}
%=====================================

Un fichier Latex est repérable par son extension ``.tex''.  Un exemple
de fichier Latex est donné sur chamilo dans le dossier TP Scilab-Latex.

\paragraph{Editer et compiler un fichier Latex} \
\begin{enumerate}
\item Pour éditer et modifier ce fichier, il suffit d'utiliser votre
  éditeur de texte préféré, par exemple :
\begin{verbatim}
$ emacs enonce.tex &
\end{verbatim}
\item Pour compiler ce fichier, taper la commande :
\begin{verbatim}
$ pdflatex enonce.tex
\end{verbatim}
\item Si la compilation s'est exécutée sans erreurs, un fichier
  ``enonce.pdf'' a été créé, à visualiser par exemple avec:
\begin{verbatim}
$ evince enonce.pdf &
\end{verbatim}
\end{enumerate}
{\it NB: Il est conseillé de garder la fenètre avec le pdf active (en
insérant un $\&$ à la fin de la commande). Aprés chaque modification
puis compilation du fichier tex, le fichier pdf est mis à jour et
rafraîchi automatiquement.}

\paragraph{Structure d'un fichier Latex} \ \\
Le langage Latex comprend deux types d'éléments (en plus du texte normal) : des commandes et des environnements.

Une \textbf{commande} est identifiable par le préfixe \textbackslash , et peut comprendre des paramètres indiqués entre \textbf{\{ \}} et des options indiquées entre \textbf{[ ]}.

Un \textbf{environnement} se déclare à l'aide de deux commandes {\tt \textbackslash begin} et {\tt \textbackslash end}. Le contenu compris entre ces deux balises aura une mise en forme, ou comprendra des commandes spécifiques à l'environnement.

\ \\La première ligne d'un fichier Latex est toujours la déclaration de sa classe.

\begin{question} Repérez la classe à laquelle appartient cet énoncé. Changez cette classe pour {\tt book} et observez les changements sur le document.
\end{question}
S'en suivent, dans un ordre quelconque :
\begin{itemize}
  \item import des packages utilisés (encodage du document, langue, environnements spécifiques...)
  \item définition personnalisée de commandes, d'environnements
  \item informations relatives au document : date, auteur, titre
\end{itemize}
Tous ces éléments donnent des informations importantes pour la compilation du fichier, mais ne vont rien afficher directement sur le document pdf généré.

\paragraph{Contenu du document} \ \\
C'est l'environnement {\tt document} qui va générer le contenu du document pdf. Dans cet environnement, en plus du texte classique, on peut intégrer (entre autre) des commandes spécifiant :

\begin{itemize}
  \item la structure de document (sections)
  \item l'organisation et la mise en forme du texte
  \item l'affichage et le référencement de figures (contenant par exemple des images, ou des tableaux)
  \item du contenu spécialisé (par exemple mathématiques, chimie ou encore musique)
\end{itemize}

Certaines commandes permettent aussi de générer automatiquement un titre, une table des matières, ou une page de références bibliographique.

\begin{question} Essayez d'inclure une image dans le document .tex. Pour cela on utilisera la commande {\tt \textbackslash includegraphics} dans l'environnement {\tt figure}.
\end{question}

\paragraph{Ecrire des mathématiques} \ \\
Il existe deux principales manières d'inclure du contenu mathématique dans un document Latex : au milieu de texte, en encadrant le contenu mathématique par des balises \$, ou alors sur une ligne séparée avec l'environnement {\tt equation}.

\paragraph{Mise en page}\ \\
La mise en page du fichier se fait de manière automatique. En particulier les sections que vous aurez indiqué seront automatiquement numérotées, de même que les figures, les équations et théorèmes. Le placement des figure est aussi décidé par Latex, ce qui peut donner parfois des résultats inattendus. Deux solutions pour remédier à une figure éloignée du texte qui lui fait référence :
\begin{itemize}
  \item ajouter l'option {\tt [h]} ou {\tt [h!]} à l'environnement {\tt figure} pour forcer la position de la figure par rapport au texte
  \item laisser la figure se placer à un endroit différent du document, et ajouter une référence que l'on pourra citer dans le texte. Pour cela, il faut utiliser la commande {\tt \textbackslash label} dans l'environnement {\tt figure}, pour associer un mot clé à votre figure. Dans votre texte, vous pouvez faire référence à cette figure en utilisant la commande {\tt \textbackslash ref} suivi du mot clé que vous lui avez attribué. Ceci fonctionne aussi pour tout ce qui est numéroté dans la structure générale du document : sections, équations, théorèmes..
\end{itemize}

Vous pouvez maintenant passer à la partie suivante du TP, et utiliser
le poly Latex au fur et à mesure des besoins que vous aurez.


%===================================================
\section{Travaux pratiques - Compte-rendu}
%===================================================

Les exercices de cette partie sont à rédiger dans le compte-rendu
Latex. Pour vous aider dans l'écriture des formules mathématiques,
vous pouvez récupérer le source du TP (le .tex) sur la page web du
stage.

\subsection{Sensibilisation à l'arithmétique machine }

Scilab, comme la plupart de langages de calcul, utilise la norme
IEEE-754 pour le stockage des nombres en mémoire. Sans rentrer dans
les détails, un nombre réel est représenté par un nombre flottant
\begin{equation*}
  x = 0.b_1b_2 \hdots b_m \, 10^e.
\end{equation*}
La partie $0.b_1b_2 \hdots b_m$ s'appelle la mantisse, $e$ est
l'exposant. Comme la mantisse n'a qu'un nombre $m$ fixé de chiffres
significatifs, Scilab ne pourra pas distinguer 2 réels au-delà de cette précision. A
titre indicatif, en Scilab la mantisse est codée sur 52 bits,
l'exposant sur 11 bits et le signe sur 1, ce qui fait qu'un réel
occupe 64 bits (8 octets) en mémoire.

En Scilab, on ne peut donc pas calculer un nombre avec une précision
arbitraire. La constante {\tt \%eps} est la distance entre le nombre
$1$ et le flottant machine qui lui est immédiatement supérieur : on
l'appelle le zéro machine, il vaut $2^{-52} \approx
2.22*10^{-16}$. Pour mieux comprendre, essayez les commandes
suivantes:
\begin{verbatim}
--> format(20)
--> 1 + %eps
--> 1 + 0.5 * %eps
\end{verbatim}

\begin{exo} Exécuter dans Scilab les commandes suivantes. Expliquer et
  commenter les résultats.
\begin{verbatim}
--> x = 1e30
--> y = 1e-8
--> z = ((y + x) - x) / y
--> w = (y + (x - x)) / y
\end{verbatim}
\end{exo}

\begin{exo} On considère la fonction $f$ définie de la manière
  suivante : pour $x \in [0,4]$ on calcule
  \begin{itemize}
  \item $y = \sqrt{\sqrt{\sqrt {\dots{\sqrt{ x}}}}}$ (128 fois)
  \item puis $f(x) = ((\dots((y^2)^2)^2\dots)^2)^2$ (128 fois)
  \end{itemize}
  Tracer la courbe de $f$ en fonction de $x$. Que
  constatez-vous ? Expliquer le résultat.
\end{exo}

\begin{exo} On cherche à calculer la valeur de l'intégrale
  \begin{equation*}
    I_{20} = \int_0^1 x^{20} e^{x} dx
  \end{equation*}
  \begin{enumerate}
  \item Donner une formule de récurrence pour calculer cette intégrale, puis évaluer $I_{20}$
    à partir de $I_0$
  \item Évaluer $I_{20}$ en utilisant le développement en série de $e^x$.
  \item Conclusions ?
  \end{enumerate}
\end{exo}

\begin{exo} Ici, on va calculer directement une approximation de
  $I_{20}$ en utilisant la méthode des rectangles.  Écrire une fonction
  {\tt I = rectangle(n)} qui calcule $I_{20}$ en utilisant $n$
  points. Vérifier que, pour $n$ suffisamment grand, on retrouve bien le
  même résultat qu'à la question 3.2.
\end{exo}


\subsection{Étude du phénomène de Gibbs}

Le phénomène de Gibbs est un phénomène oscillatoire qui s'observe
graphiquement quand on approche une fonction par sa série de
Fourier. L'exercice qui suit vous fait d'abord calculer la série de
Fourier d'une fonction ``simple'', puis tracer cette série tronquée
aux premiers termes.

\paragraph*{Rappel sur les séries de Fourier}
Soit $f:\mathbb{R}\mapsto\mathbb{C}$ une fonction $T$ periodique, la série
de Fourier de $f$ s'écrit comme une combinaison linéaire de fonctions sinusoïdales :
\begin{equation}
	f(t) = \sum_{n=-\infty}^{\infty} c_{n}(f)\exp\left(2i\pi\frac{n}{T}t\right),
\end{equation}
où les coefficients $c_{n}(f)$ de $f$ sont appelé \emph{coefficients de Fourier}
et sont définis comme suit :
\begin{equation}
	c_{n}(f) = \frac{1}{T}\int_{-T/2}^{T/2} f(t)\exp\left(-2i\pi\frac{n}{T}t\right)dt.
\end{equation}
Dans le cas de fonctions $f:\mathbb{R}\mapsto\mathbb{R}$, quelques simplifications
peuvent être faites, la série de Fourier peut s'écrire comme suit :
\begin{equation}
	f(t) = a_{0}(f) + \sum_{n=1}^{\infty} a_{n}(f)\cos\left(2\pi\frac{n}{T}t\right) + b_{n}(f)\sin\left(2\pi\frac{n}{T}t\right),
\end{equation}
où les coefficients $a_{n}(f)$, sont définis comme suit :
\begin{equation}
	\left\{\begin{aligned}
		a_{0}(f) &= \frac{1}{T}\int_{-T/2}^{T/2}f(t)dt \\
		a_{n}(f) &= \frac{2}{T}\int_{-T/2}^{T/2}f(t)\cos\left(2\pi\frac{n}{T}t\right)dt \\
		b_{n}(f) &= \frac{2}{T}\int_{-T/2}^{T/2}f(t)\sin\left(2\pi\frac{n}{T}t\right)dt \\
	\end{aligned}\right..
\end{equation}

\begin{exo} (théorique) Soit la fonction 1-périodique $f$, définie
  sur $[-\frac{1}{2}, \frac{1}{2}[$ par
  \begin{equation}
    \label{f}
    f(x) = \left\{\begin{array}{lll}
        -1 & \text{pour} & -\frac{1}{2} \le x \le 0 \,,\\
        1 & \text{pour} & 0< x < \frac{1}{2} \,.
      \end{array}\right.
  \end{equation}
  Calculer la série de Fourier de $f$ et expliquer en
  quels points $x$ la formule ci-dessous est valide :
  \begin{equation}
    \label{Ff}
    f(x) = \frac{4}{\pi} \sum_{n=0}^{+\infty} \frac{\sin(2(2n+1)\pi x)}{2n+1} \,.
  \end{equation}
\end{exo}

\begin{exo} {\it (pratique)} Écrire un programme Scilab traçant la
  série de Fourier tronquée de la fonction $f$ (formule \eqref{f}).
  Concrètement, reprendre la formule \eqref{Ff} et calculer la somme
  partielle des {\tt (Ntermes+1)}-premiers termes, pour une valeur de
  {\tt Ntermes} entrée au clavier :
  \begin{lstlisting}[language=scilab]
    Ntermes = input('Entrer le nombre de termes')
  \end{lstlisting}
  Pour la boucle de contrôle :
  \begin{lstlisting}[language=scilab]
    for i = 0:Ntermes
        S = S + ...
    end
  \end{lstlisting}
  Quel phénomène observez-vous sur le graphe? Comment varie-t-il avec
  {\tt Ntermes}? Que pouvez-vous en conclure sur la convergence de la
  série de Fourier?
\end{exo}


\subsection{Théorème de Gerschg\"orin}

\paragraph*{(petit) Rappel sur les valeurs propres et vecteurs propres}
Soit $A$ une matrice, les valeurs propres $\lambda$ et les vecteurs propres
$v$ de $A$ respectent l'égalité :
\begin{equation}
	(A - \lambda I)v,
\end{equation}
où $I$ est la matrice identité.

\begin{exo} Dans cet exercice, nous nous intéressons au théorème suivant :\\

  Soit $A$ une matrice carrée d'ordre $N$. Les valeurs propres de
  $A$ appartiennent à l'union des $N$ disques $D_k$ du plan
  complexe, soit : $\lambda \in \bigcup\limits_{k=1}^N D_k$, où
  $D_k$, appelé disque de Gerschg\"orin est défini par:
  \begin{equation*}
    D_k = \{ z \in \C : |z - a_{kk}| \leq \Lambda_k =
    \sum\limits_{j=1,j\neq k}^N |a_{kj}| \} \,.
  \end{equation*}
  \begin{enumerate}
  \item Démontrer ce théorème.
  \item Écrire un programme Scilab permettant de visualiser les disques
    de Gerschg\"orin dans le plan complexe (pour tracer un cercle tracer
    deux demi-cercles en utilisant la commande {\tt plot2d}).
  \item Vérifier que les valeurs propres de la matrice $A$ suivante sont toutes
  dans l'union des disques de Gerschg\"orin de $A$ :
  \begin{equation*}
    A = \left( \begin{array}{cccc}
      1+i & i & 2 \\
      3 & 2+i & 1\\
      1 & i   & 6
    \end{array} \right).
  \end{equation*}
  \item On appelle matrice \`a diagonale strictement dominante, une
    matrice telle que
    \begin{equation*}
      \forall i \quad \sum\limits_{k\neq i} |a_{ik}| < |a_{ii}|.
    \end{equation*}
    Démontrer, en utilisant les disques de Gerschg\"orin que toute matrice
    à diagonale strictement dominante est inversible.
  \end{enumerate}
\end{exo}

% =============
\end{document}
% =============
