\documentclass[a4paper,11pt]{article}

\usepackage[utf8x]{inputenc}
\usepackage[T1]{fontenc}
\usepackage[francais]{babel}
\usepackage{amsmath,amssymb}
\usepackage{fullpage}
\usepackage{xspace}
\usepackage{verbatim}
\usepackage{graphicx}
\usepackage{listings}
\usepackage{mathrsfs}
\usepackage[usenames,dvipsnames]{color}
\usepackage{url}

\lstset{basicstyle=\small\tt,
  keywordstyle=\bfseries\color{Orchid},
  stringstyle=\it\color{Tan},
  commentstyle=\it\color{LimeGreen},
  showstringspaces=false}

\newtheorem{exo}{Question}

\newcommand{\dx}{\,dx}
\newcommand{\ito}{,\dotsc,}
\newcommand{\R}{\mathbb{R}}
\newcommand{\C}{\mathbb{C}}
\newcommand{\N}{\mathbb{N}}
\newcommand{\Poly}[1]{\mathcal{P}_{#1}}
\newcommand{\abs}[1]{\left\lvert#1\right\rvert}
\newcommand{\norm}[1]{\left\lVert#1\right\rVert}
\newcommand{\pars}[1]{\left(#1\right)}
\newcommand{\bigpars}[1]{\bigl(#1\bigr)}
\newcommand{\set}[1]{\left\{#1\right\}}

\title{TP m�thodes num�riques : Compte-rendu}
\author{Billy Ndihokubwayo et Michel Yoeung (ENSIMAG - 1A - Groupe 2)}
\date{Avril 2018}

\begin{document}

\maketitle

%======================================
\section{Introduction.}
%======================================

%======================================
\section{Simulation de la collision de deux billes par le sch�ma d�Euler explicite.}
%======================================

\begin{exo} \ \\
\begin{equation}
\frac{d^{2}x_{1}}{dt^{2}} = -({x_{1}-x_{2}})^{\frac{3}{2}}_{+}
\end{equation} \ 
\begin{equation}
\frac{d^{2}x_{2}}{dt^{2}} = ({x_{1}-x_{2}})^{\frac{3}{2}}_{+}
\end{equation} \ 
\begin{equation}
H = \frac{v_{1}^{2}}{2} + \frac{v_{2}^{2}}{2} + \frac{2}{5}({x_{1}-x_{2}})^{\frac{5}{2}}_{+}
\end{equation} \ \\
On a donc :
\begin{equation}
\begin{align}
\frac{dH}{dt} & = \frac{1}{2}\frac{dv_{1}^{2}}{dt} + \frac{1}{2}\frac{dv_{2}^{2}}{dt} + 
(\frac{2}{5}(x_{1}-x_{2})^{\frac{5}{2}}_{+})' \ \\
& = \frac{2}{2}v_{1}\frac{dv_{1}}{dt} + \frac{2}{2}v_{2}\frac{dv_{2}}{dt} + ({v_{1}-v_{2}})_{+}(x_{1}-x_{2})^{\frac{3}{2}}_{+} \ \\
& = -v_{1}(v_{1}-v_{2})^{\frac{3}{2}}_{+} + v_{2}(v_{1}-v_{2})^{\frac{3}{2}}_{+} + ({v_{1}-v_{2}})_{+}(x_{1}-x_{2})^{\frac{3}{2}}_{+} \ \\
& = -(v_{1}-v_{2})_{+}(x_{1}-x_{2})^{\frac{3}{2}}_{+} + (v_{1}-v_{2})_{+}(x_{1}-x_{2})^{\frac{3}{2}}_{+} \ \\
& = 0
\end{align}
\end{equation} \ \\
\end{exo}
\begin{exo} \ \\ \\
\begin{equation}
\begin{align}
Y =
\begin{pmatrix} 
x_{1} \\
x_{2} \\
v_{1} \\
v_{2}
\end{pmatrix}
\mapsto
\begin{pmatrix} 
F_{1}(Y) \\
F_{2}(Y) \\
F_{3}(Y) \\
F_{4}(Y)
\end{pmatrix}
avec \frac{dY}{dt}=F(Y)
& \Rightarrow 
Y =
\begin{pmatrix} 
x_{1}(t) \\
x_{2}(t) \\
v_{1}(t) \\
v_{2}(t)
\end{pmatrix}
\mapsto
\begin{pmatrix} 
\frac{dx_{1}}{dt}(t) \\
\frac{dx_{2}}{dt}(t) \\
\frac{dv_{1}}{dt}(t) \\
\frac{dv_{2}}{dt}(t)
\end{pmatrix} \ \\
& \Rightarrow 
Y =
\begin{pmatrix} 
x_{1}(t) \\
x_{2}(t) \\
v_{1}(t) \\
v_{2}(t)
\end{pmatrix}
\mapsto
\begin{pmatrix} 
v_{1}(t) \\
v_{2}(t) \\
-(x_{1}-x_{2})^{\frac{3}{2}}_{+} \\
+(x_{1}-x_{2})^{\frac{3}{2}}_{+}
\end{pmatrix} \ \\
\end{align}
\end{equation} \ \\
\end{exo}

\begin{exo}
(Voir script Scilab)
\end{exo}

\begin{exo}
(Voir graphe) \ \\ \\
Pour $h_{1} = \frac{1}{10}$ : \ \\
$v_{1}(0) = 1$ \ \\
$v_{2}(0) = 0$ \ \\
$v_{1,finale} = -0.13$ \ \\
$v_{2,finale} = 1.13$ \ \\ \\
Pour $h_{2} = \frac{1}{100}$ : \ \\
$v_{1}(0) = 1$ \ \\
$v_{2}(0) = 0$ \ \\
$v_{1,finale} = -0.012$ \ \\
$v_{2,finale} = 1.012$ \ \\ \\
Pour $h_{3} = \frac{1}{1000}$ : \ \\
$v_{1}(0) = 1$ \ \\
$v_{2}(0) = 0$ \ \\
$v_{1,finale} = -0.0011$ \ \\
$v_{2,finale} = 1.0011$ \ \\ \\
On peut donc remarquer que $v_{2,finale} > v_{1}(0)$ pour les trois valeurs de $h$. \ \\
Ces r�sultats ne me semble pas r�alistes �tant donn� que la deuxi�me bille
aura au finale une plus grande vitesse que la vitesse initiale de la premi�re 
bille l'ayant touch�. Ce qui vaudrait dire que si que le mouvement des billes
ne s'arr�terait jamais. \ \\
Et si on prend $h \to 0$, on remaque que $v_{2,finale} \to v_{1}(0)$.

\end{exo}


%======================================
\section{Simulation de la collision de n billes par le sch�ma d�Euler implicite.}
%======================================

\begin{exo} \ \\ \\
On a :
\left\{
\begin{array}{ll}
m_{1}\frac{d^{2}x_{1}}{dt^{2}} = -(x_{1}-x_{2})^{\frac{3}{2}}_{+} \ \\ \\
m_{i}\frac{d^{2}x_{i}}{dt^{2}} = (x_{i-1}-x_{i})^{\frac{3}{2}}_{+}-(x_{i}-x_{i+1})^{\frac{3}{2}}_{+} \; pour \; 2\leq i\leq n-1 \ \\ \\
m_{n}\frac{d^{2}x_{n}}{dt^{2}} = -(x_{n-1}-x_{n})^{\frac{3}{2}}_{+}
\end{array}
\right \ \\ \\
qui \; mod�lise \; la \; collision \; de \; n \; billes \ \\ \\
et \; $H = \sum_{i=1}^{n}(\frac{1}{2}m_{i}(\frac{dx_{i}}{dt})^{2}) + \frac{2}{5}\sum_{i=1}^{n-1}(x_{i}-x_{i+1})^{\frac{5}{2}}_{+}$ \ \\ \\
On a donc :
\begin{equation}
\begin{align}
\frac{dH}{dt} & = \frac{d(\sum_{i=1}^{n}(\frac{1}{2}m_{i}(\frac{dx_{i}}{dt})^{2}))}{dt} + \frac{d(\frac{2}{5}\sum_{i=1}^{n-1}(x_{i}-x_{i+1})^{\frac{5}{2}}_{+})}{dt} \ \\
& = \frac{2}{2}\sum_{i=1}^{n}(\frac{1}{2}m_{i}(\frac{dx_{i}}{dt}\frac{d^{2}x_{i}}{dt^{2}})) + \frac{2}{5}*\frac{5}{2}\sum_{i=1}^{n-1}((v_{i}-v_{i+1})_{+}(x_{i}-x_{i+1})^{\frac{3}{2}}_{+}) \ \\
& = -\sum_{i=1}^{n}(v_{i}m_{i}\frac{d^{2}x_{i}}{dt^{2}}) + \sum_{i=1}^{n-1}(v_{i}(x_{i}-x_{i+1})^{\frac{3}{2}}_{+} - v_{i+1}(x_{i}-x_{i+1})^{\frac{3}{2}}_{+}) \ \\
& = -v_{1}(x_{1}-x_{2})^{\frac{3}{2}}_{+} + \sum_{i=2}^{n-1}(v_{i}((x_{i-1}-x_{i})^{\frac{3}{2}}_{+} - (x_{i}-x_{i+1})^{\frac{3}{2}}_{+})) + v_{n}(x_{n-1}-x_{n})^{\frac{3}{2}}_{+} + \ \\
& \;\;\;\;\; v_{1}(x_{1}-x_{2})^{\frac{3}{2}}_{+} - v_{2}(x_{1}-x_{2})^{\frac{3}{2}}_{+} +
\sum_{i=2}^{n-2}(v_{i}(x_{i}-x_{i+1})^{\frac{3}{2}}_{+} - v_{i+1}(x_{i}-x_{i+1})^{\frac{3}{2}}_{+}) + \ \\
& \;\;\;\;\; v_{n}(x_{n-1}-x_{n})^{\frac{3}{2}}_{+} - v_{n}(x_{n-1}-x_{n})^{\frac{3}{2}}_{+} \ \\
& = \sum_{i=2}^{n-1}(v_{i}(x_{i-1}-x_{i})^{\frac{3}{2}}_{+} - v_{i}(x_{i}-x_{i+1})^{\frac{3}{2}}_{+}) - v_{2}(x_{1}-x_{2})^{\frac{3}{2}}_{+} + \ \\ & \;\;\;\;\; \sum_{i=2}^{n-2}(v_{i}(x_{i}-x_{i+1})^{\frac{3}{2}}_{+} - v_{i+1}(x_{i}-x_{i+1})^{\frac{3}{2}}_{+}) + v_{n-1}(x_{n-1}-x_{n})^{\frac{3}{2}}_{+} \ \\
& = 0 \;\;\; car \; les \; termes \; s'annulent
\end{align}
\end{equation}
\end{exo}

\begin{exo} \ \\ \\
On a :
M =
\begin{pmatrix} 
m_{1} & 0 & ... & 0 \\
0 & m_{2} & ... & ... \\
... & ... & ... & 0 \\
0 & ... & 0 & m_{n}
\end{pmatrix}
et Mx =
\begin{pmatrix} 
m_{1}x_{1} \\
m_{2}x_{2} \\
... \\
m_{n}x_{n}
\end{pmatrix} \ \\ \\
Et en se basant sur le sch�ma d'Euler implicite : \ \\ \\
\begin{pmatrix} 
m_{1} & 0 & ... & 0 \\
0 & m_{2} & ... & ... \\
... & ... & ... & 0 \\
0 & ... & 0 & m_{n}
\end{pmatrix}
\begin{pmatrix} 
x_{1}^{(k+1)}-2x_{1}^{(k)}+x_{1}^{(k-1)} \\
x_{2}^{(k+1)}-2x_{2}^{(k)}+x_{2}^{(k-1)} \\
... \\
x_{n}^{(k+1)}-2x_{n}^{(k)}+x_{n}^{(k-1)}
\end{pmatrix}
- h^{2}f(x^{(k+1)}) \ \\ \\ \\
=
\begin{pmatrix} 
m_{1}(x_{1}^{(k+1)}-2x_{1}^{(k)}+x_{1}^{(k-1)}) \\
m_{2}(x_{2}^{(k+1)}-2x_{2}^{(k)}+x_{2}^{(k-1)}) \\
... \\
m_{n}(x_{n}^{(k+1)}-2x_{n}^{(k)}+x_{n}^{(k-1)})
\end{pmatrix}
- h^{2}f(x^{(k+1)}) \ \\ \\ \\
=
h^{2}
\begin{pmatrix} 
\frac{m_{1}(x_{1}^{(k+1)}-2x_{1}^{(k)}+x_{1}^{(k-1)})}{h^{2}} \\
\frac{m_{2}(x_{2}^{(k+1)}-2x_{2}^{(k)}+x_{2}^{(k-1)})}{h^{2}} \\
... \\
\frac{m_{n}(x_{n}^{(k+1)}-2x_{n}^{(k)}+x_{n}^{(k-1)})}{h^{2}}
\end{pmatrix}
- h^{2}f(x^{(k+1)}) \;\;\;\;\; avec \; le \; vecteur \; qui \; designe \; $f(x^{(k+1)})$
\ \\ \\ \\
= h^{2}f(x^{(k+1)})-h^{2}f(x^{(k+1}) \ \\ \\
= 0
\ \\ \\
Donc \; le \; sch�ma \; d'Euler \; implicite \; s'�crit \; bien \; : \ \\ \\
M(x^{(k+1)}-2x^{(k)}+x^{(k-1)})-h^{2}f(x^{(k+1)}) = 0
\end{exo}

\begin{exo}
(Voir script Scilab)
\end{exo}

\begin{exo}
(Voir script Scilab)
\end{exo}

\begin{exo}
(Voir graphe) \ \\ \\
L'�nergie m�canique diminue en effet avec le temps. \ \\
Et plus on prend h petit, plus l'�nergie m�canique diminue lentement

\end{exo}

\begin{exo}
(Voir graphe) \ \\ \\
Avec le graphe en niveaux de gris (Sgrayplot), on peut constater que les forces de contacts entre deux billes vont en s'amplifiant jusqu'� un certain moment puis diminuenent et viennent � s'annuler une fois pour toutes. Toutefois ce cycle ne se remarque pas au m�me instant pour tous les couples de billes voisines. En effet par exemple, une fois que les forces de contact entre la premi�re et la deuxi�me bille commencent � diminuer, le cycle d�bute pour la deuxi�me et la troisi�me bille et ainsi de suite.
\end{exo}

\begin{exo}
(Voir graphe) \ \\ \\
Les r�sultats obtenus pour $m=0.5$ montrent que l'�volution des forces de contacts entre $2$ billes ne suit plus le cycle observ� sur les r�sultats pour $m=1$.  En effet pour $m=0.5$, les forces de contacts entre 2 billes donn�es peuvent augmenter puis s'annuler puis par la suite augmenter encore(ceci peut se r�p�ter plusieurs fois) ce qui n'�tait pas le cas pour $m=1$ o� les forces de contact entre $2$ billes donn�es ne pouvaient r�augmenter une fois apr�s avoir �t� annul�es.
\end{exo}

\begin{exo}
(Voir graphe) \ \\ \\
Il faut choisir $m=0.5$ pour avoir une vitesse d'ejection minimale.
\end{exo}

\begin{exo} \ \\ \\
On rappelle :
M =
\begin{pmatrix} 
m_{1} & 0 & ... & 0 \\
0 & m_{2} & ... & ... \\
... & ... & ... & 0 \\
0 & ... & 0 & m_{n}
\end{pmatrix} \ \\ \\ \\
On a :
\begin{pmatrix} 
\frac{m_{1}(u_{1}^{(k+1)}-2u_{1}^{(k)}+u_{1}^{(k-1)})}{h^{2}} \\
... \\
\frac{m_{i}(u_{i}^{(k+1)}-2u_{i}^{(k)}+u_{i}^{(k-1)})}{h^{2}} \\
... \\
\frac{m_{n}(u_{n}^{(k+1)}-2u_{n}^{(k)}+u_{n}^{(k-1)})}{h^{2}}
\end{pmatrix}
=
\begin{pmatrix} 
u_{2}^{(k+1)}-2u_{1}^{(k+1)}+f_{1}((k+1)h) \\
... \\
u_{i+1}^{(k+1)}-2u_{i}^{(k+1)}+u_{i-1}^{(k+1)} \\
... \\
u_{n-1}^{(k+1)}-2u_{n}^{(k+1)}
\end{pmatrix} \ \\ \\ \\
\Rightarrow
\begin{pmatrix} 
m_{1}(u_{1}^{(k+1)}-2u_{1}^{(k)}+u_{1}^{(k-1)})} \\
... \\
m_{i}(u_{i}^{(k+1)}-2u_{i}^{(k)}+u_{i}^{(k-1)})} \\
... \\
m_{n}(u_{n}^{(k+1)}-2u_{n}^{(k)}+u_{n}^{(k-1)})}
\end{pmatrix}
=
h^{2}
\begin{pmatrix} 
u_{2}^{(k+1)}-2u_{1}^{(k+1)}+f_{1}((k+1)h) \\
... \\
u_{i+1}^{(k+1)}-2u_{i}^{(k+1)}+u_{i-1}^{(k+1)} \\
... \\
u_{n-1}^{(k+1)}-2u_{n}^{(k+1)}
\end{pmatrix} \ \\ \\ \\
\Rightarrow
\begin{pmatrix} 
m_{1}u_{1}^{(k+1)} \\
... \\
m_{i}u_{i}^{(k+1)} \\
... \\
m_{n}u_{n}^{(k+1)}
\end{pmatrix}
+
M
\begin{pmatrix} 
-u_{1}^{(k)} \\
... \\
-u_{i}^{(k)} \\
... \\
u_{n}^{(k)}
\end{pmatrix}
-
M
\begin{pmatrix} 
h\sqrt{1}^{(k)} \\
... \\
h\sqrt{i}^{(k)} \\
... \\
h\sqrt{n}^{(k)}
\end{pmatrix}
=
h^{2}
\begin{pmatrix} 
u_{2}^{(k+1)}-2u_{1}^{(k+1)}+f_{1}((k+1)h) \\
... \\
u_{i+1}^{(k+1)}-2u_{i}^{(k+1)}+u_{i-1}^{(k+1)} \\
... \\
u_{n-1}^{(k+1)}-2u_{n}^{(k+1)}
\end{pmatrix} \ \\ \\ \\
\Rightarrow
\begin{pmatrix} 
m_{1}u_{1}^{(k+1)} \\
... \\
m_{i}u_{i}^{(k+1)} \\
... \\
m_{n}u_{n}^{(k+1)}
\end{pmatrix}
-
\begin{pmatrix} 
m_{1}(u_{1}^{(k)}+h\sqrt{1}^{(k)}) \\
... \\
m_{i}(u_{i}^{(k)}+h\sqrt{i}^{(k)}) \\
... \\
m_{n}(u_{n}^{(k)}+h\sqrt{n}^{(k)})
\end{pmatrix}
=
-h^{2}
\begin{pmatrix} 
1-f_{1}((k+1)h) & -1 & 0 & ... & ... & 0 \\
-1 & 2 & -1 & 0 & ... & ... \\
0 & -1 & 2 & ... & ... & 0 \\
... & ... & ... & ... & ... & -1 \\
0 & ... & ... & 0 & -1 & 2
\end{pmatrix} \ \\ \\ \\
\begin{pmatrix} 
u_{1}^{(k+1)} \\
... \\
u_{i}^{(k+1)} \\
... \\
u_{n}^{(k+1)}
\end{pmatrix} \ \\ \\
avec \;\; $Mu^{(k+1)}=$
\begin{pmatrix} 
m_{1}u_{1}^{(k+1)} \\
... \\
m_{i}u_{i}^{(k+1)} \\
... \\
m_{n}u_{n}^{(k+1)}
\end{pmatrix}
, $b^{(k)}=$
\begin{pmatrix} 
m_{1}(u_{1}^{(k)}+h\sqrt{1}^{(k)}) \\
... \\
m_{i}(u_{i}^{(k)}+h\sqrt{i}^{(k)}) \\
... \\
m_{n}(u_{n}^{(k)}+h\sqrt{n}^{(k)})
\end{pmatrix}, \ \\ \\ \\
$D=$
\begin{pmatrix} 
1-f_{1}((k+1)h) & -1 & 0 & ... & ... & 0 \\
-1 & 2 & -1 & 0 & ... & ... \\
0 & -1 & 2 & ... & ... & 0 \\
... & ... & ... & ... & ... & -1 \\
0 & ... & ... & 0 & -1 & 2
\end{pmatrix}
et $u^{(k+1)}=$
\begin{pmatrix} 
u_{1}^{(k+1)} \\
... \\
u_{i}^{(k+1)} \\
... \\
u_{n}^{(k+1)}
\end{pmatrix} 
\ \\ \\ \\
\Rightarrow
$Mu^{(k+1)}+h^{2}Du^{(k+1)} = b^{(k)}$ \ \\ \\
\Rightarrow
$(M+h^{2}-D)u^{(k+1)} = b^{(k)}$ \ \\ \\
\Rightarrow
$Au^{(k+1)} = b^{(k)}$ avec $A = M+h^{2}-D$ \ \\ \\
De plus, $A^{t} = (M+h^{2}D) = M^{t}+h^{2}D^{t} = M+h^{2}D = A$ \ \\
Donc A est bien une matrice sym�trique et tridiagonale.
\end{exo}

\begin{exo} \ \\

\end{exo}












\end{document}


